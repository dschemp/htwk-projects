\setcounter{section}{0}
\section{Pflichtenheft}

\pagebreak

\begin{tabularx}{\textwidth}{| P | X |}
    \hline
    Dokument                             & Pflichtenheft                               \\ \hline
    Projektleiter                        & \highlight{Vorname Nachname}                \\ \hline
    Projekt                              & \highlight{Projektname}                     \\ \hline
    \multicolumn{2}{c}{} \\ \hline
    Verantwortlich für dieses Dokument   & \highlight{Vorname Name}                    \\ \hline
    Erstellt am\footnotemark             & \highlight{TT.MM.JJJJ}                      \\ \hline
    Zuletzt geändert\footnotemark        & \highlight{TT.MM.JJJJ}                      \\ \hline
    Bearbeitungsstand\footnotemark & 
        {
            \noindent % Ich habe keinen Plan, wie ich den Platz links von der kleinen Tabelle hier weg bekomme, ohne dieses Padding auch für alle anderen Zellen in der großen Tabelle gleich mitzunuken.
            \begin{tabular}{| c | l |}
                \highlight{X} & In Bearbeitung  \\ \hline
                  & Vorgelegt       \\ \hline
                  & Fertig gestellt
            \end{tabular}
        }
    \\ \hline
    Dateiablage                          & \highlight{Dateiablage}                     \\ \hline
\end{tabularx}

% Footnotes müssen außerhalb von floating environments wie table oder figure gemacht werden. Ansonsten kann man problemlos \footnote{} nutzen
\footnotetext{Datum, an dem das Dokument erstmalig erzeugt wurde}
\footnotetext{Datum der letzten Änderung}
\footnotetext{Beachten Sie die Hinweise in der Vorlesung bzw. im Wiki.}

\pagebreak
