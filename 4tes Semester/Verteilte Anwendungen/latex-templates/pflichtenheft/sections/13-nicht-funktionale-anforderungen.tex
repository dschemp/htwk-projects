\subsection{Nicht funktionale Anforderungen}

\textit{
    Beschreiben Sie die nicht funktionalen Anforderungen der zukünftigen Software. Wenn die funktionalen Anforderungen die qualitativen Anforderungen erfassen, finden sich hier i.d.R. die quantitativen Anforderungen wieder. Darüber hinaus können sich Anforderungen aus der Organisation, rechtlichen Bedingungen oder Compliances ergeben. Die folgenden Überschriften sollen Ihnen helfen, die nicht funktionalen Anforderungen zu identifizieren.
}

\textit{
    Jede Anforderung sollte messbar sein, wie z. B. „Das System sollte für 80\% der Nutzer innerhalb von 3 Sek eine Antwort liefern“. Überlegen Sie bereits hier, wie die spätere Messung (Review, Inspektion, etc.) stattfinden soll und binden Sie das QS-Team ein.
}

\subsubsection{Zuverlässigkeit}

\highlight{Text}


\subsubsection{Benutzbarkeit}

\highlight{Text}


\subsubsection{Effizienz}

\highlight{Text}


\subsubsection{Änderbarkeit}

\highlight{Text}


\subsubsection{Übertragbarkeit}

\highlight{Text}


\subsubsection{Migration}

\highlight{Text}


\subsubsection{Wartbarkeit}

\highlight{Text}


\subsubsection{Testbarkeit}

\highlight{Text}


\subsubsection{IT-Sicherheit}

\highlight{Text}


\subsubsection{Randbedingungen}

\highlight{Text}
