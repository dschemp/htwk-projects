\subsection{Funktionale Anforderungen}

\textit{
    Funktionale Anforderungen beschreiben die Fähigkeiten eines Systems, die ein Anwender erwartet, um mit Hilfe des Systems ein fachliches Problem zu lösen. Die Anforderungen werden aus den zu unterstützenden Geschäftsprozessen und den Ablaufbeschreibungen zur Nutzung des Systems abgeleitet.
}

\textit{
    Die Beschreibung der funktionalen Anforderungen erfolgt in Form von Anwendungsfällen (Use Cases). Ein Anwendungsfall beschreibt dabei einen konkreten, fachlich in sich geschlossenen Teilvorgang. Die Gesamtheit der Anwendungsfälle definiert das Systemverhalten.
}

\highlight{Text}

\subsubsection{Geschäftsprozesse und Geschäftsanwendungsfälle}

\textit{Erstellen Sie nun die Beschreibung der funktionalen Anforderungen entsprechend Anwendungsfälle/UML Use Cases. In einem Anwendungsfall wird die Funktionalität des Softwaresystems dargestellt, z.B. „Die App bietet die Möglichkeit, sich sicher anzumelden.“. Selbsterklärend sollten die Anwendungsfälle zu den Nutzeranforderungen passen, gleichwohl können sie darüber hinausgehen. }

\begin{itemize}
    \item Use-Case-Modell und Aktivitätsdiagramme 
    \item Übrige fachliche Anforderungen und Regeln
\end{itemize}
