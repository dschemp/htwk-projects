\subsection{Festlegungen zur Kommunikation}

\subsubsection{Analyse der benötigten Kommunikation}

\textit{
    In der folgenden Tabelle wird festgehalten, welchen Informationsbedarf die verschiedenen Projektbeteiligten haben und mit welchem Medium in welcher Frequenz die Information verteilt wird. 
    Ferner wird festgelegt, ob eine Hol- (Pull) oder eine Bringschuld (Push) existiert.
}

\textit{
    Oft wird vergessen, wer sich wann die Informationen abholt oder geliefert bekommt.
    Klären Sie gleich zu Beginn des Projekts die Regeln und profitieren Sie später von einer effektiven Kommunikation.
}

\begin{tabularx}{\textwidth}{| >{\scriptsize}c | >{\scriptsize}c | >{\scriptsize\centering}X | >{\scriptsize}c | >{\scriptsize}c |}
    \hline
    \rowcolor{lightgray} 
    Name / Rolle            & Art der Info & Frequenz & Medium & Push / Pull \\ \hline
    Projektleiter & Projektstatusbericht & \highlight{Zu jedem MS (s. Projektauftrag)} & \highlight{E-M@il} & Push \\ \hline
    \highlight{<Name/Rolle>} & \highlight{<Art der Info>} & \highlight{<Frequenz>} & \highlight{<Medium>} & \highlight{<Push/Pull>} \\ \hline
    \highlight{<Name/Rolle>} & \highlight{<Art der Info>} & \highlight{<Frequenz>} & \highlight{<Medium>} & \highlight{<Push/Pull>} \\ \hline
    \highlight{<Name/Rolle>} & \highlight{<Art der Info>} & \highlight{<Frequenz>} & \highlight{<Medium>} & \highlight{<Push/Pull>} \\ \hline
\end{tabularx}

\subsubsection{Planung von Jour Fixes}

Der erste Jour Fixe findet statt am \highlight{Datum}, die weiteren im \highlight{x}-wöchentlichen Abstand.


\subsubsection{Projektstatusberichte}

\textit{Entfällt.}


\subsubsection{Gesamtdokumentation}

\textit{Die Abgabe der prüfungsrelevanten Dokumentation erfolgt als PDF/A Export.}


\subsubsection{Abstimmungsprozesse, Konfliktmanagement und Eskalationsstrategie}

Entscheidungen werden auf folgender Basis getroffen \textit{[Zutreffendes auswählen]}:

\highlight{Die einfache Mehrheit ist ausreichend / Eine Zwei-Drittel-Mehrheit ist erforderlich / Es ist generell der Konsens (Einstimmigkeit) herzustellen / <Rollen> haben ein Vetorecht.}

Kann kein Ergebnis zwischen den im Konfliktfall direkt Beteiligten erreicht werden, greift folgende Eskalationsleiter: 

\highlight{Rolle 1 $\rightarrow$ Rolle 2 $\rightarrow$ Rolle 3}

Der nicht gelöste Konflikt wird jeweils eine Stufe weiter geleitet, bis schließlich eine Entscheidung getroffen wird.
