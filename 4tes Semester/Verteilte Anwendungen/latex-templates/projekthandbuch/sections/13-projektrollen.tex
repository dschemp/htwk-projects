\subsection{Projektrollen}

\textit{
    Im Projekt erfolgen die Rollenbesetzung und die Zuordnung von Rechten und Pflichten wie folgt. Beachten Sie, dass diese Tabelle zur Bewertung des Projekts und damit hinsichtlich der Vergabe der Noten im Modul herangezogen werden kann. Abweichungen von den vorgegebenen Rollen sind zu begründen. Eine Rolle kann mehreren Personen zugeordnet werden und eine Person kann mehrere Rollen ausüben. Weitere Rollen können bei Bedarf ergänzt werden, z.B. im Zusammenhang mit agilen Vorgehensmodellen.
}

\begin{tabularx}{\textwidth}{| >{\scriptsize}p{6cm} | >{\scriptsize}X |}
    \hline
    \rowcolor{lightgray} 
    Rolle                                                       & Besetzung \\ \hline
    Projektmanager (Projektleiter)                              & \highlight{<Name>} \\ \hline
    Projektleitungsassistenz                                    & \highlight{<Name>} \\ \hline
    Dokumentationsverantwortlicher                              & \highlight{<Name>} \\ \hline
    Qualitätsmanager                                            & \highlight{<Name>} \\ \hline
    Projektkontroller                                           & \highlight{<Name>} \\ \hline
    Risikomanager                                               & \highlight{<Name>} \\ \hline
    Verantwortlicher für Konfigurations- und Versionsmanagement & \highlight{<Name>} \\ \hline
    Kommunikationsverantwortlicher                              & \highlight{<Name>} \\ \hline
    Anforderungsmanager                                         & \highlight{<Name>} \\ \hline
    Designer                                                    & \highlight{<Name>} \\ \hline
    Architekt                                                   & \highlight{<Name>} \\ \hline
    Entwickler                                                  & \highlight{<Name>} \\ \hline
\end{tabularx}
