\subsection*{Änderungsverzeichnis}

\textit{
    Im Änderungsverzeichnis werden alle Bearbeitungen des Dokuments eingetragen. Es sollte darauf geachtet werden, dass die letzte Änderung mit dem Datum auf dem Deckblatt weiter oben übereinstimmt. Änderungen sollen mit jeder neuen Version dokumentiert werden (Minor/Major).
}

\begin{tabularx}{\textwidth}{| >{\scriptsize}c | >{\scriptsize}c | >{\scriptsize}c | >{\scriptsize}c | >{\scriptsize\centering}X | >{\scriptsize}c | >{\scriptsize}c |}
    \hline
    \rowcolor{lightgray}
    \multicolumn{3}{| c |}{\scriptsize{Änderung}} & \multicolumn{4}{c |}{} \\ \hline
    \rowcolor{lightgray}
    Nr. & Datum & Version & Geänderte Kapitel & Beschreibung der Änderung & Autor & Zustand \\ \hline
    \highlight{1} & \highlight{TT.MM.JJJJ} & \highlight{0.1} & \highlight{Alle} & \highlight{Initale Produkterstellung} & \highlight{Name} & \highlight{In Bearbeitung} \\ \hline
    &       &         &                   &                           &       &         \\ \hline
\end{tabularx}
