\subsection{Festlegungen zur inhaltlichen Ausführung}

\subsubsection{Planung der Anforderungsanalyse}
Der erste Workshop zur Anforderungserhebung findet statt am \highlight{Datum}.

\subsubsection*{Teilnehmer:}
\begin{tabularx}{\textwidth}{| >{\scriptsize}p{6cm} | >{\scriptsize}X |}
    \hline
    \rowcolor{lightgray} 
    Name               & Rolle \\ \hline
    \highlight{<Name>} & Anforderungsmanager \\ \hline
    \highlight{<Name>} & \highlight{<Rolle>} \\ \hline
    \highlight{<Name>} & \highlight{<Rolle>} \\ \hline
\end{tabularx}

Die weiteren Workshops finden im Abstand von \highlight{X} Wochen statt.


\subsubsection{Planung des Anforderungsmanagements}
Im weiteren Lebenszyklus des Projektes werden die Kundenanforderungen mit den zugehörigen Arbeitsprodukten (d. h. Anforderung $\rightarrow$ Design $\rightarrow$ SW-Modul $\rightarrow$ Test etc.) verlinkt.

\textit{Dokumentieren und begründen Sie hier Ihr Vorgehen.}

\textbf{Verantwortlich ist der Anforderungsanalytiker.}


\subsubsection{Planung des Vorgehensmodells}
Im weiteren Lebenszyklus des Projektes wird nach einem ausgewählten Vorgehensmodell:
\highlight{V-Modell, Spiral-Modell, XP, Kanban, RUP die Projektorganisation iterativ / inkrementell / agil / prototypisch} entwickelt.

\textit{Die Auswahl des Vorgehens wird wie folgt \textbf{begründet}.
Informationen zu den Vorgehensmodellen sind im Skript, im Lernmodul und im Internet zu finden.
Die entsprechenden Standards, Normen und Richtlinien sind einzuhalten.
Änderungen bei der Durchführung sind zu begründen!}

\textbf{Verantwortlich ist der Projektleiter.}

\subsubsection{Methodisches Vorgehen in den Bereichen Safety / Security und Datenschutz}
\textit{Informationen zu den Standards und Richtlinien finden Sie im Skript, im Lernmodul, im Intranet und im Internet.}

\textbf{Verantwortlich ist der QS-Manager.}
