\subsection{Festlegungen zum operativen Projektmanagement}

\textit{
    Für das operative Projektmanagement werden an dieser Stelle die grundlegenden Projektregeln aufgestellt. Sie bilden die Basis für alle Prozesse des Projekts und gelten für alle Beteiligten.
}

\subsubsection{Projektinfrastruktur}
\paragraph{Administrative Infrastruktur}

Für Meetings, Jour Fixe etc. stehen folgende (auch virtuellen) Räume zur Verfügung: 
\begin{enumerate}
    \item \highlight{Raum 1}
    \item \highlight{Raum 2}
    \item \highlight{...}
\end{enumerate}

\paragraph{Technische Infrastruktur}

\begin{tabularx}{\textwidth}{| >{\scriptsize}X | >{\scriptsize}X |}
    \hline
    \rowcolor{lightgray} 
    Aufgabe                       & Werkzeug \\ \hline
    Projektplanung                & \highlight{z. B. MS Project} \\ \hline
    Versionsverwaltung            & GitHub \\ \hline
    \highlight{Weitere Aufgaben} & \highlight{Weitere Werkzeuge} \\ \hline
\end{tabularx}


\subsubsection{Regeln zur Erstellung und Lenkung von PM-Dokumenten}

\textit{
    Regeln Sie die Erstellung und Verteilung von Dokumenten im Projekt klar und deutlich. Die folgende Tabelle soll Sie dabei unterstützen. Die Einträge sind eine Hilfestellung – fügen Sie hier ihre eigenen Regeln ein.
}

\begin{tabularx}{\textwidth}{| >{\scriptsize}X | >{\scriptsize}l | >{\scriptsize}l | >{\scriptsize}p{7cm} |}
    \hline
    \rowcolor{lightgray} 
    Dokument / Produkt & Erstellt von & Gegeben an & Wann \\ \hline
    Protokoll eines Termins & Protokollführer & alle Teilnehmer & max. 2 Werktage nach Besprechung \\ \hline
    QS-Bericht & QS-Manager & Projektleiter & Nach Vorgabe des Projektplans, i. d. R. 2 Werktage vor Fertigstellung des Projektstatusbericht \\ \hline
    Prüfberichte & Prüfer & QS-Manager & entsprechend Prüfplan – abhängig von der Definition der Prüfobjekte \\ \hline
    Projektstatusbericht & Projektleiter & Auftraggeber & In der Regel 7 Tage vor dem nächsten Meilenstein. \\ \hline
\end{tabularx}
