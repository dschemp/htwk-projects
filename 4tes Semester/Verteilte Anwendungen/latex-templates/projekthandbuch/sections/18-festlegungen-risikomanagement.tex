\subsection{Festlegungen zum Risikomanagement}

\subsubsection{Risikokategorien festlegen}

\textit{Folgende Risikokategorien werden im Projekt verwendet:}
\highlight{
    \begin{itemize}
        \item Akzeptanz
        \item Arbeitsbasis, Projektauftrag
        \item Terminliche Risiken
        \item Technische Risiken
        \item Organisatorische Risiken (Prozess, Struktur)
        \item Kapazitive Risiken, Ressourcen und Skills
        \item Sicherheitsrisiken
        \item Qualitative Risiken
        \item Vorgehensmodell und eingesetzte Tools
    \end{itemize}
}


\subsubsection{Erfassung und Verfolgung von Risiken}
Das erste Meeting zur Erfassung der Risiken findet statt am \highlight{Datum}.

\subsubsection*{Teilnehmer:}
\begin{tabularx}{\textwidth}{| >{\scriptsize}p{6cm} | >{\scriptsize}X |}
    \hline
    \rowcolor{lightgray} 
    Name               & Rolle \\ \hline
    \highlight{<Name>} & Anforderungsmanager \\ \hline
    \highlight{<Name>} & \highlight{<Rolle>} \\ \hline
    \highlight{<Name>} & \highlight{<Rolle>} \\ \hline
\end{tabularx}
Die weiteren Meetings zur Verfolgung der Risiken finden im Abstand von \highlight{X} Wochen statt.
