%!TEX spellcheck = de_DE
%!TEX TS-program = xelatex
%!TEX encoding = UTF-8 Unicode

\documentclass[a4paper, 11pt]{article}

\input{preamble.sty}

\usepackage{color, soul, colortbl}
\usepackage{tocloft}
\usepackage{setspace}

\graphicspath{ {./imgs/} } % Autocomplete

% höhrere Zellen
\renewcommand{\arraystretch}{1.5}
% Spalte mit fester Breite und in Hellgrau
\newcolumntype{P}{>{\columncolor{lightgray}}p{4cm}}
% "Benötigt Änderung"
\newcommand{\highlight}[1]{\textcolor{yellow!70!blue}{#1}}
% vierte Aufzählungsebene
\setcounter{secnumdepth}{4}
\let\origparagraph\paragraph
\renewcommand{\paragraph}[1]{\origparagraph{#1}\mbox{}\\}

% ToC Einstellungen
\renewcommand\cftsecafterpnum{\vskip 5pt} % for spacing after each entry
\renewcommand\cftsubsecafterpnum{\vskip 5pt} % for spacing after each entry
\renewcommand\cftsubsubsecafterpnum{\vskip 5pt} % for spacing after each entry

% Daten zum Dokument
%\title{<< Title >>}
%\author{<< Author >>}

% Dokument
\begin{document}
% --- Insert text here

% "Deckblatt"
\input{sections/00-projekthandbuch.tex}

% Änderungsverzeichnis
\subsection*{Änderungsverzeichnis}

\begin{tabularx}{\textwidth}{| >{\scriptsize}c | >{\scriptsize}c | >{\scriptsize}c | >{\scriptsize}c | >{\scriptsize\centering}X | >{\scriptsize}c | >{\scriptsize}c |}
    \hline
    \rowcolor{lightgray}
    \multicolumn{3}{| c |}{\scriptsize{Änderung}} & \multicolumn{4}{c |}{} \\ \hline
    \rowcolor{lightgray}
    Nr. & Datum & Version & Geänderte Kapitel & Beschreibung der Änderung & Autor & Zustand \\ \hline
    \highlight{1} & \highlight{TT.MM.JJJJ} & \highlight{0.1} & \highlight{Alle} & \highlight{Initale Produkterstellung} & \highlight{Name} & \highlight{In Bearbeitung} \\ \hline
    &       &         &                   &                           &       &         \\ \hline
\end{tabularx}


% Prüfverzeichnis
\subsection*{Prüfverzeichnis}

Die folgende Tabelle zeigt einen Überblick über alle Prüfungen – sowohl Eigenprüfungen wie
auch Prüfungen durch eigenständige Qualitätssicherung – des vorliegenden Dokumentes.

\begin{tabularx}{\textwidth}{| >{\scriptsize}c | >{\scriptsize}c | >{\scriptsize\centering}X | >{\scriptsize}c | >{\scriptsize}c |}
    \hline
    \rowcolor{lightgray} 
    Datum                    & Geprüfte Version & Anmerkung & Prüfer & Neuer Produktzustand \\ \hline
    \highlight{TT.MM.JJJJ}  &                  &           &        &                      \\ \hline
                             &                  &           &        &                      \\ \hline
\end{tabularx}

\pagebreak


% Inhaltsverzeichnis
\input{sections/03-toc.tex}

% Ziel, Umfang und Abgrenzung des Projekts
\subsection{Ziel, Umfang und Abgrenzung des Projekts}

\textit{Fassen Sie Ziel und Auftrag des Projekts kurz aus Ihrer Sicht zusammen!}

\highlight{Beschreibung}


% Abweichungen von Standards
\subsection{Abweichungen von Standards}

\textit{
    Sollten Sie von dem vorgegebenen Vorgehen oder einem anderen Standard abweichen – hinsichtlich der Softwareentwicklung oder auch des Projektvorgehens – dokumentieren und begründen Sie hier Ihre Entscheidung!
}

\begin{tabularx}{\textwidth}{| >{\scriptsize}l | >{\scriptsize}X |}
    \hline
    \rowcolor{lightgray} 
    Abweichung                         & Begründung              \\ \hline
    \highlight{Abweichung, Referenz}  & \highlight{Begründung} \\ \hline
    \highlight{Abweichung, Referenz}  & \highlight{Begründung} \\ \hline
\end{tabularx}


% Meilensteinplan
\subsection{Meilensteinplan}

\textit{
    Jedem Projekt liegt ein Vorgehen mit wesentlichen Meilensteinen zugrunde. Einige Meilensteine werden Ihnen zu Semesterbeginn vorgegeben (Projektauftrag). Ergänzen Sie diese Liste mit eigenen Terminen! Übertragen Sie die Meilensteine in das Meilensteinregister! Das folgende Beispiel enthält einen typischen Meilensteinplan für ein Entwicklungsprojekt.
}

\begin{tabularx}{\textwidth}{| >{\scriptsize}X | >{\scriptsize}l | >{\scriptsize}p{4cm} |}
    \hline
    \rowcolor{lightgray} 
    Name                                                                  & Geplantes Enddatum      & Ergebnisse und Ziele \\ \hline
    \highlight{Vorarbeiten abgeschlossen}                                & \highlight{tt.mm.yyyy} & \highlight{Erläuterung} \\ \hline
    \highlight{Design, Konzeption: UML und Pflichtenheft abgeschlossen}  & \highlight{tt.mm.yyyy} & \highlight{Erläuterung} \\ \hline
    \highlight{Implementierung abgeschlossen}                            & \highlight{tt.mm.yyyy} & \highlight{Erläuterung} \\ \hline
    \highlight{Präsentation abgeschlossen}                               & \highlight{tt.mm.yyyy} & \highlight{Erläuterung} \\ \hline
    \highlight{Abgabe}                                                   & \highlight{tt.mm.yyyy} & \highlight{Erläuterung} \\ \hline
\end{tabularx}


% Projektrollen
\subsection{Projektrollen}

\textit{
    Im Projekt erfolgen die Rollenbesetzung und die Zuordnung von Rechten und Pflichten wie folgt. Beachten Sie, dass diese Tabelle zur Bewertung des Projekts und damit hinsichtlich der Vergabe der Noten im Modul herangezogen werden kann. Abweichungen von den vorgegebenen Rollen sind zu begründen. Eine Rolle kann mehreren Personen zugeordnet werden und eine Person kann mehrere Rollen ausüben. Weitere Rollen können bei Bedarf ergänzt werden, z.B. im Zusammenhang mit agilen Vorgehensmodellen.
}

\begin{tabularx}{\textwidth}{| >{\scriptsize}p{6cm} | >{\scriptsize}X |}
    \hline
    \rowcolor{lightgray} 
    Rolle                                                       & Besetzung \\ \hline
    Projektmanager (Projektleiter)                              & \highlight{<Name>} \\ \hline
    Projektleitungsassistenz                                    & \highlight{<Name>} \\ \hline
    Dokumentationsverantwortlicher                              & \highlight{<Name>} \\ \hline
    Qualitätsmanager                                            & \highlight{<Name>} \\ \hline
    Projektkontroller                                           & \highlight{<Name>} \\ \hline
    Risikomanager                                               & \highlight{<Name>} \\ \hline
    Verantwortlicher für Konfigurations- und Versionsmanagement & \highlight{<Name>} \\ \hline
    Kommunikationsverantwortlicher                              & \highlight{<Name>} \\ \hline
    Anforderungsmanager                                         & \highlight{<Name>} \\ \hline
    Designer                                                    & \highlight{<Name>} \\ \hline
    Architekt                                                   & \highlight{<Name>} \\ \hline
    Entwickler                                                  & \highlight{<Name>} \\ \hline
\end{tabularx}


% Festlegungen zum operativen Projektmanagement
\subsection{Festlegungen zum operativen Projektmanagement}

\textit{
    Für das operative Projektmanagement werden an dieser Stelle die grundlegenden Projektregeln aufgestellt. Sie bilden die Basis für alle Prozesse des Projekts und gelten für alle Beteiligten.
}

\subsubsection{Projektinfrastruktur}
\paragraph{Administrative Infrastruktur}

Für Meetings, Jour Fixe etc. stehen folgende (auch virtuellen) Räume zur Verfügung: 
\begin{enumerate}
    \item \highlight{Raum 1}
    \item \highlight{Raum 2}
    \item \highlight{...}
\end{enumerate}

\paragraph{Technische Infrastruktur}

\begin{tabularx}{\textwidth}{| >{\scriptsize}X | >{\scriptsize}X |}
    \hline
    \rowcolor{lightgray} 
    Aufgabe                       & Werkzeug \\ \hline
    Projektplanung                & \highlight{z. B. MS Project} \\ \hline
    Versionsverwaltung            & GitHub \\ \hline
    \highlight{Weitere Aufgaben} & \highlight{Weitere Werkzeuge} \\ \hline
\end{tabularx}


\subsubsection{Regeln zur Erstellung und Lenkung von PM-Dokumenten}

\textit{
    Regeln Sie die Erstellung und Verteilung von Dokumenten im Projekt klar und deutlich. Die folgende Tabelle soll Sie dabei unterstützen. Die Einträge sind eine Hilfestellung – fügen Sie hier ihre eigenen Regeln ein.
}

\begin{tabularx}{\textwidth}{| >{\scriptsize}X | >{\scriptsize}l | >{\scriptsize}l | >{\scriptsize}p{7cm} |}
    \hline
    \rowcolor{lightgray} 
    Dokument / Produkt & Erstellt von & Gegeben an & Wann \\ \hline
    Protokoll eines Termins & Protokollführer & alle Teilnehmer & max. 2 Werktage nach Besprechung \\ \hline
    QS-Bericht & QS-Manager & Projektleiter & Nach Vorgabe des Projektplans, i. d. R. 2 Werktage vor Fertigstellung des Projektstatusbericht \\ \hline
    Prüfberichte & Prüfer & QS-Manager & entsprechend Prüfplan – abhängig von der Definition der Prüfobjekte \\ \hline
    Projektstatusbericht & Projektleiter & Auftraggeber & In der Regel 7 Tage vor dem nächsten Meilenstein. \\ \hline
\end{tabularx}


% Festlegungen zum Änderungsmanagement
\subsection{Festlegungen zum Änderungsmanagement}

\textit{
    Da es sich um ein Lehr- und Lernprojekt handelt, wird das Änderungsmanagement nicht explizit adressiert. Änderungen im Projektauftrag sind nur im Ausnahmefall zulässig und durch den betreuenden Dozenten freizugeben.
}


% Festlegungen zur inhaltlichen Ausführung
\subsection{Festlegungen zur inhaltlichen Ausführung}

\subsubsection{Planung der Anforderungsanalyse}
Der erste Workshop zur Anforderungserhebung findet statt am \highlight{Datum}.

\subsubsection*{Teilnehmer:}
\begin{tabularx}{\textwidth}{| >{\scriptsize}p{6cm} | >{\scriptsize}X |}
    \hline
    \rowcolor{lightgray} 
    Name               & Rolle \\ \hline
    \highlight{<Name>} & Anforderungsmanager \\ \hline
    \highlight{<Name>} & \highlight{<Rolle>} \\ \hline
    \highlight{<Name>} & \highlight{<Rolle>} \\ \hline
\end{tabularx}

Die weiteren Workshops finden im Abstand von \highlight{X} Wochen statt.


\subsubsection{Planung des Anforderungsmanagements}
Im weiteren Lebenszyklus des Projektes werden die Kundenanforderungen mit den zugehörigen Arbeitsprodukten (d. h. Anforderung $\rightarrow$ Design $\rightarrow$ SW-Modul $\rightarrow$ Test etc.) verlinkt.

\textit{Dokumentieren und begründen Sie hier Ihr Vorgehen.}

\textbf{Verantwortlich ist der Anforderungsanalytiker.}


\subsubsection{Planung des Vorgehensmodells}
Im weiteren Lebenszyklus des Projektes wird nach einem ausgewählten Vorgehensmodell:
\highlight{V-Modell, Spiral-Modell, XP, Kanban, RUP die Projektorganisation iterativ / inkrementell / agil / prototypisch} entwickelt.

\textit{Die Auswahl des Vorgehens wird wie folgt \textbf{begründet}.
Informationen zu den Vorgehensmodellen sind im Skript, im Lernmodul und im Internet zu finden.
Die entsprechenden Standards, Normen und Richtlinien sind einzuhalten.
Änderungen bei der Durchführung sind zu begründen!}

\textbf{Verantwortlich ist der Projektleiter.}

\subsubsection{Methodisches Vorgehen in den Bereichen Safety / Security und Datenschutz}
\textit{Informationen zu den Standards und Richtlinien finden Sie im Skript, im Lernmodul, im Intranet und im Internet.}

\textbf{Verantwortlich ist der QS-Manager.}


% Festlegungen zum Konfigurationsmanagement
\subsection{Festlegungen zum Konfigurationsmanagement}

\textit{Im Bereich Projektinfrastruktur (siehe 2.5.1) wird festgelegt, welches Werkzeug benutzt wird.}

\subsubsection{Regeln zur Versionierung / Entwicklungsumgebung}
\textit{
    Für die Versionsverwaltung ist GitHub zu verwenden. Informationen zur Versionsverwaltung finden Sie im Skript und im Lernmodul. Eine geeignete Entwicklungsumgebung ist auszuwählen und die Auswahl ist zu begründen. Die entsprechenden Standards, Normen und Richtlinien sind einzuhalten. Weitere Informationen zu Entwicklungsumgebungen finden Sie im Intranet und im Internet.
}

\textbf{Verantwortlich ist der Konfigurationsmanagementverantwortliche.}

\subsubsection{Datensicherung und -archivierung}

Die Datensicherung erfolgt \highlight{...}


% Festlegungen zum Risikomanagement
\subsection{Festlegungen zum Risikomanagement}

\subsubsection{Risikokategorien festlegen}

\textit{Folgende Risikokategorien werden im Projekt verwendet:}
\highlight{
    \begin{itemize}
        \item Akzeptanz
        \item Arbeitsbasis, Projektauftrag
        \item Terminliche Risiken
        \item Technische Risiken
        \item Organisatorische Risiken (Prozess, Struktur)
        \item Kapazitive Risiken, Ressourcen und Skills
        \item Sicherheitsrisiken
        \item Qualitative Risiken
        \item Vorgehensmodell und eingesetzte Tools
    \end{itemize}
}


\subsubsection{Erfassung und Verfolgung von Risiken}
Das erste Meeting zur Erfassung der Risiken findet statt am \highlight{Datum}.

\subsubsection*{Teilnehmer:}
\begin{tabularx}{\textwidth}{| >{\scriptsize}p{6cm} | >{\scriptsize}X |}
    \hline
    \rowcolor{lightgray} 
    Name               & Rolle \\ \hline
    \highlight{<Name>} & Anforderungsmanager \\ \hline
    \highlight{<Name>} & \highlight{<Rolle>} \\ \hline
    \highlight{<Name>} & \highlight{<Rolle>} \\ \hline
\end{tabularx}
Die weiteren Meetings zur Verfolgung der Risiken finden im Abstand von \highlight{X} Wochen statt.


% Festlegungen zur Kommunikation
\subsection{Festlegungen zur Kommunikation}

\subsubsection{Analyse der benötigten Kommunikation}

\textit{
    In der folgenden Tabelle wird festgehalten, welchen Informationsbedarf die verschiedenen Projektbeteiligten haben und mit welchem Medium in welcher Frequenz die Information verteilt wird. 
    Ferner wird festgelegt, ob eine Hol- (Pull) oder eine Bringschuld (Push) existiert.
}

\textit{
    Oft wird vergessen, wer sich wann die Informationen abholt oder geliefert bekommt.
    Klären Sie gleich zu Beginn des Projekts die Regeln und profitieren Sie später von einer effektiven Kommunikation.
}

\begin{tabularx}{\textwidth}{| >{\scriptsize}c | >{\scriptsize}c | >{\scriptsize\centering}X | >{\scriptsize}c | >{\scriptsize}c |}
    \hline
    \rowcolor{lightgray} 
    Name / Rolle            & Art der Info & Frequenz & Medium & Push / Pull \\ \hline
    Projektleiter & Projektstatusbericht & \highlight{Zu jedem MS (s. Projektauftrag)} & \highlight{E-M@il} & Push \\ \hline
    \highlight{<Name/Rolle>} & \highlight{<Art der Info>} & \highlight{<Frequenz>} & \highlight{<Medium>} & \highlight{<Push/Pull>} \\ \hline
    \highlight{<Name/Rolle>} & \highlight{<Art der Info>} & \highlight{<Frequenz>} & \highlight{<Medium>} & \highlight{<Push/Pull>} \\ \hline
    \highlight{<Name/Rolle>} & \highlight{<Art der Info>} & \highlight{<Frequenz>} & \highlight{<Medium>} & \highlight{<Push/Pull>} \\ \hline
\end{tabularx}

\subsubsection{Planung von Jour Fixes}

Der erste Jour Fixe findet statt am \highlight{Datum}, die weiteren im \highlight{x}-wöchentlichen Abstand.


\subsubsection{Projektstatusberichte}

\textit{Entfällt.}


\subsubsection{Gesamtdokumentation}

\textit{Die Abgabe der prüfungsrelevanten Dokumentation erfolgt als PDF/A Export.}


\subsubsection{Abstimmungsprozesse, Konfliktmanagement und Eskalationsstrategie}

Entscheidungen werden auf folgender Basis getroffen \textit{[Zutreffendes auswählen]}:

\highlight{Die einfache Mehrheit ist ausreichend / Eine Zwei-Drittel-Mehrheit ist erforderlich / Es ist generell der Konsens (Einstimmigkeit) herzustellen / <Rollen> haben ein Vetorecht.}

Kann kein Ergebnis zwischen den im Konfliktfall direkt Beteiligten erreicht werden, greift folgende Eskalationsleiter: 

\highlight{Rolle 1 $\rightarrow$ Rolle 2 $\rightarrow$ Rolle 3}

Der nicht gelöste Konflikt wird jeweils eine Stufe weiter geleitet, bis schließlich eine Entscheidung getroffen wird.


% Abkürzungsverzeichnis
\subsection{Glossar, Abkürzungsverzeichnis}

\begin{tabularx}{\textwidth}{| >{\scriptsize}p{4cm} | >{\scriptsize}X |}
    \hline
    \rowcolor{lightgray} 
    Begriff, Abkürzung               & Erläuterung \\ \hline
    & \\ \hline
    & \\ \hline
    & \\ \hline
\end{tabularx}


% ---

\end{document}
